\documentclass{amsart}
\usepackage{amsmath}
\usepackage{amsfonts}
\usepackage{amssymb}
\usepackage{amsthm}
\usepackage{graphicx}
\usepackage{tikz}
\usepackage{tkz-euclide}
\usetikzlibrary{intersections}
\usetikzlibrary{
  knots,
  hobby,
  decorations.pathreplacing,
  shapes.geometric,
  calc
}
%\usetikzlibrary{arrows,chains,matrix,positioning,scopes,knots,shapes,spy,snakes}


\theoremstyle{plain}
\newtheorem{theorem}{Theorem}[section]
\newtheorem{lemma}[theorem]{Lemma}
\newtheorem{proposition}[theorem]{Proposition}
\newtheorem{corollary}{Corollary}[section]

\theoremstyle{definition}
\newtheorem{definition}{Definition}[section]
\newtheorem{conjecture}{Conjecture}[section]
\newtheorem{example}{Example}[section]
\newtheorem{claim}{Claim}[section]

\theoremstyle{remark}
\newtheorem{exercise}{Exercise}
\newtheorem*{note}{Note}

\begin{document}

\title{Topology, Geometry and Gauge Fields}
\author{Danny B}
\date{February 20, 2016}


\section{Basic Concepts in Topology}


\begin{definition}
A Topological Space blah $(\mathcal{X}, \mathcal{T})$ is a set $\mathcal{X}$ with a family $\mathcal{T}$ of subsets of $\mathcal{T}$ such that:
\begin{enumerate}
 \item $\mathcal{X} \in \mathcal{T}$
\end{enumerate}
\end{definition}


\end{document}
